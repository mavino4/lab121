%
% Compilation of the examples from the TikZ-UML manual, v. 1.0b (2013-03-01)
% http://www.ensta-paristech.fr/~kielbasi/tikzuml/index.php?lang=en
%
\documentclass[a4paper,12pt]{article}

\usepackage[T1]{fontenc}
\usepackage[utf8x]{inputenc}
\usepackage[french]{babel}
\usepackage{fullpage}

\usepackage{tikz-uml}

\sloppy
\hyphenpenalty 10000000

\title{Package Example: TikZ UML Diagrams}
\author{Nicolas Kielbasiewicz}

\begin{document}

\maketitle

\section{Class Diagram}

\newpage

\begin{tikzpicture}
\umlclass{Estudiante}
{ - CI : int \\
-  nombre : string \\
- np : int \\
- Prestados: int[]\\
}
{
+ Estudiante() : void\\
+ Estudiante(int,string) : void\\
+ writeArch(BinaryWriter) : void\\
+ readArch(BinaryReader) : void\\
+ prestarX(int) : void\\
+ devolverX(int cod) : bool\\
+ getCI() : int\\
+ getNombre() : string\\
}

\umlclass[x=8,y=0]{ArchEst}
{ - nombre : string\\
- route : string\\
}
{ + ArchEst(string) : void\\
+ crearNuevo() : void\\
+ addE(Estudiante) : void\\
+ listar() : LinkedList<Estudiante>\\
+ getEstudiante(int): Estudiante\\
+ eliminar(int) : bool\\
+ getRoute() : string}

\umlclass[x=0, y=-10]{Recurso}
{- codR : int\\
- nombre : string\\
- estado : bool\\
- CIprest : int}
{
+ Recurso(int , string) : void\\
+ Recurso() : void\\
+ prestarR() : void\\
+ writeArch(BinaryWriter) : void\\
+ readArch(BinaryReader) : void\\
+ getCodR() : int\\
+ getNombreR() : string\\
+ getEstadoR() : bool\\
+ setEstado(bool) : void\\
+ getCiPrest() : int \\
+ setCiPrest(int) : void
}

\umlclass[x=8,y=-10]{ArchRec}
{ - nombre : string\\
- route : string\\
}
{ + ArchRec(string) : void\\
+ crearNuevo() : void\\
+ addR(Recurso) : void\\
+ listar() : LinkedList<Recurso>\\
+ getrecurso(int): Recurso\\
+ eliminar(int) : bool\\
+ getRoute() : string}

\end{tikzpicture}\\
\begin{tikzpicture}
\umlclass{MainForm : Form}
{ - archE : ArchEst\\
- archR : ArchRec\\
}{ + MainForm() : void \\
+ MainFormLoad() : void \\
+ AddRecClick() : void\\
+ PrestarClick() : void\\
+ DevolverClick() : void\\
+ QuienTieneYClick() : void\\
+ AddEstClick() : void\\
+ QueDebeXClick() : void\\
- FormClosed() : void\\
+ listar() : void
}
\end{tikzpicture}
\\
\begin{tikzpicture}
\umlclass{AddRecursos : Form}
{ - archR : ArchRec\\
}{ + AddRecursos() : void \\
+ AdicionarClick() : void\\
}
\end{tikzpicture}
\\
\begin{tikzpicture}
\umlclass{Prestar : Form}
{  - archE : ArchEst\\
- archR : ArchRec\\
}{ + Prestar() : void \\
+ PrestarClick() : void\\
}
\end{tikzpicture}
\\
\begin{tikzpicture}
\umlclass{Devolver : Form}
{  - archE : ArchEst\\
- archR : ArchRec\\
}{ + Devolver() : void \\
+ DevolverClick() : void\\
}
\end{tikzpicture}
\\
\begin{tikzpicture}
\umlclass{QuienTieneY : Form}
{  - archE : ArchEst\\
- archR : ArchRec\\
}{ + QuienTieneY() : void \\
+ ConsularClick() : void\\
+ CerrarClick() : void
}
\end{tikzpicture}
\\
\begin{tikzpicture}
\umlclass{AddEst : Form}
{  - archE : ArchEst\\
}{ + AddEst() : void \\
+ AddEstClick() : void\\
}
\end{tikzpicture}
\\
\begin{tikzpicture}
\umlclass{QueDebeX : Form}
{  - archE : ArchEst\\
- archR : ArchRec\\
}{ + QueDebeX() : void \\
+ ConsularClick() : void\\
+ CerrarClick() : void
}
\end{tikzpicture}
\end{document}